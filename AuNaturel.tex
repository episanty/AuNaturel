



\documentclass[twoside,british,a4paper,twocolumn]{article}
%%% Au Naturel
%%% A LaTeX article template. See https://github.com/episanty/TheYoungManTheStation for more.
%%% © Emilio Pisanty (2016), available under the MIT license.



\usepackage{amsmath}
\usepackage{amsfonts}
\usepackage{amssymb,upref}
\usepackage{tgheros} %% Main sans-serif font
\usepackage{tgtermes} %% Main serif font
\usepackage{anyfontsize}
\usepackage{xcolor}
\usepackage{graphicx}
\usepackage{enumitem}
\usepackage{subfig}
\usepackage{siunitx}


%\usepackage{flushend} %% don't use - there's some incompatibility somewhere
\usepackage{balance}

% Use accented characters
\usepackage[utf8]{inputenc}
\usepackage[T1]{fontenc}

%% slightly prettier date
\usepackage{isodate}
\cleanlookdateon





%% Fancy first letter
\usepackage{lettrine}
\pdfmapfile{=montserrat.map} 
\renewcommand{\LettrineFont}{
  \fontfamily{iwonal}\fontsize{42}{42}\selectfont %% Iwona Light
%  \fontfamily{Montserrat-LF}\fontseries{l} \fontsize{42}{42}\selectfont %% Montserrat light
%  \fontfamily{Montserrat-LF}\fontseries{l} \fontsize{41}{41}\selectfont %% Montserrat extralight ?
%  \fontfamily{Chivo-TLF}\fontseries{l} \fontsize{42}{42}\selectfont %% Chivo light
  \color[rgb]{0,0.25,0.8}
  }
%% use as \lettrine[lines=3]{T}{\,} he


%% Mainly useful at draft stage
%\usepackage{showframe}
\usepackage{lipsum}

%\usepackage{draftwatermark}
%\SetWatermarkText{draft}
%\SetWatermarkScale{1}



%%%% math font
\usepackage{newtxmath}
%\usepackage{mathptmx} % Greek letters slightly too heavy
%% - http://www.gust.org.pl/projects/e-foundry/tg-math/download/index_html#Bonum_Math
%% - https://md.ekstrandom.net/blog/2014/07/tex-modern-fonts/
%% instead?




\usepackage{geometry}
\geometry{reset,ignoreall,
  textheight=253mm,
  textwidth=175mm,
  bottom=21mm,
  inner=17.5mm,
  footskip=8mm,
  headsep=5mm,
  headheight=10pt
  }
%\setlength{\parindent}{0em}
%\setlength{\parskip}{0.6\baselineskip}


\usepackage{fancyhdr}
\renewcommand{\headrule}{}
\renewcommand{\footrule}{}
%
\pagestyle{fancy}
\fancyhead{}\fancyfoot{}
\fancyhead[LE]{ \footnotesize\sffamily\upshape \shortauthor}
\fancyhead[RO]{ \footnotesize\sffamily\upshape \shorttitle}
\fancyfoot[LE,RO]{\sffamily\bfseries\upshape \thepage}
%
\fancypagestyle{firstpagestyle}
\fancyhead{}\fancyfoot{}
\fancyfoot[LE,RO]{\sffamily\bfseries\upshape \thepage}

\usepackage{titlesec}
\titleformat{\section}[block]{\bfseries\upshape\sffamily\boldmath}{}{0.em}{}
\titlespacing*{\section}{0pt}{0.8em plus 0ex minus 0ex}{0em plus 0.ex}



\usepackage{titling}
\setlength{\droptitle}{-12mm}
\pretitle{\noindent\begin{minipage}{\textwidth} \raggedright\huge\bfseries\upshape\sffamily\boldmath
\color[rgb]{0,0,0.8}
}
\posttitle{\end{minipage} \vskip 0.6em}
\preauthor{\noindent \begin{minipage}{\textwidth} \large\mdseries\sffamily}
\postauthor{
  \end{minipage} 
  \vskip 0.4em
  {
   \sffamily
   \color{black!80}
   \noindent \small
   \address
   \par %\vskip -1.em
   \authoremail
   }
   \par \vskip 0.4em
  }
\predate{ \noindent \hspace{-0.4em} \sffamily\small}
\postdate{\vskip 0.0em}


% Figure captions
\usepackage{caption}
\DeclareCaptionFont{cfs}{\fontsize{8.5}{10.25}\selectfont}
\DeclareCaptionLabelSeparator{vline}{\;|\;}
\captionsetup{
 labelsep=vline, 
 font={sf,cfs}, 
 labelfont={cfs,bf},
 belowskip=-12pt
% belowskip=0pt
 }
%\captionsetup{labelsep=vline, font={sf,cfs}, labelfont={cfs,bf}, belowskip=-12pt}
\newcommand{\figurecaption}[2]{\caption[#1]{\textbf{#1.} #2}}



%\makeatletter
%\edef\orig@output{\the\output}
%\output{\setbox\@cclv\vbox{\unvbox\@cclv\vspace{0pt plus 20pt}}\orig@output}
%\makeatother




\usepackage{tcolorbox}
\definecolor{abstractboxcolor}{cmyk}{0.1,0,0,0}
\newtcolorbox{abstractbox}{
  arc=0pt,
  boxrule=0pt,
  colback=abstractboxcolor,
  boxsep=0.5em,
  left=0pt, right=0pt, bottom=0pt, top=0pt,
  width=\columnwidth
}
%
%%% To avoid some bad box warnings, as per tex.se/q/62296/
\makeatletter
 \def\@textbottom{\vskip \z@ \@plus 1pt}
 \let\@texttop\relax
\makeatother
%
\renewenvironment{abstract}{
   \noindent
   \begin{minipage}{\textwidth}
   \upshape\sffamily \bfseries
   \fontsize{9}{11.5}\selectfont
  }{
   \end{minipage} 
   \vskip 2.0em
  }


%%% Bibliography
\usepackage[numbers,sort&compress]{natbib}

\makeatletter \def\NAT@def@citea{\def\@citea{\NAT@separator\,}} \makeatother % reduce spacing inside [1, 2].
\setlength{\bibsep}{0pt plus 0.3ex} %reduce inter-line spacing in references.

%%% Make bibtex more forgiving with underfull hboxes (as per tex.se/q/10924):
\usepackage{etoolbox}
\apptocmd{\sloppy}{\hbadness 10000\relax}{}{}

%%% Specified referencers
\newcommand{\citer}[1]{Ref.~\citealp{#1}}
\newcommand{\citers}[1]{Refs.~\citealp{#1}}
\newcommand{\reffig}[1]{Fig.~\ref{#1}}






\usepackage[%
  bookmarks=true,
  colorlinks,
  linkcolor=blue,
  urlcolor=blue,
  citecolor=blue,
  plainpages=false,
  pdfpagelabels,
  final,
  breaklinks=true
]{hyperref}
\hypersetup{
pdftitle={Au Naturel: a LaTeX article template}, 
pdfauthor={Emilio Pisanty},
pdfkeywords={latex, article, class}
}

\title{Au Naturel: a LaTeX article template}
\newcommand\shorttitle{Au Naturel}

\author{Emilio Pisanty\textsuperscript{1$\,*$}}
\newcommand\shortauthor{E. Pisanty}

\newcommand\address{\textsuperscript{1}ICFO -- The Institute of Photonic Sciences, Barcelona, Spain}
\newcommand\authoremail{$^*$author@email.com}

\date{ \today }




%%%%% Own commands
%
\usepackage{physics}
%
\renewcommand{\d}{\ensuremath{\textrm{d}}}
\renewcommand{\Re}{\operatorname{Re}}
\renewcommand{\Im}{\operatorname{Im}}
%\newcommand{\vb}[1]{\mathbf{#1}}
  \newcommand{\vbe}{\vb{E}}
  \newcommand{\vbr}{\vb{r}}
  
  \newcommand{\tve}{\tilde{\vb{E}}}

\newcommand{\ue}[1]{\hat{\vb{e}}_{#1}}


\newcommand{\noheadersection}[1]{%
  \par\refstepcounter{section}% Increase section counter
  \sectionmark{#1}% Add section mark (header)
  \addcontentsline{toc}{section}{\protect\numberline{\thesection}#1}% Add section to ToC
}


\usepackage{textcomp}

\begin{document}

\twocolumn[
\begin{@twocolumnfalse}

\maketitle
\thispagestyle{firstpagestyle}

\begin{abstract}
%\textit{Au Naturel} is a LaTeX template built on top of the standard \verb|article| class.
Au Naturel is a LaTeX template built on top of the standard article class, roughly emulating some characteristics of the Nature Publishing Group journals.
{
\vskip 0.3em
\normalfont \sffamily 
\textsf{\textcopyright}
\footnotesize 
Emilio Pisanty (2018), available under the MIT license.
}
\end{abstract}

\vspace{-2mm}

\end{@twocolumnfalse}
]
%\fontsize{9}{11}\selectfont



\section{Introduction}
\lettrine[lines=3, lhang=0.15]{T}{\:} his
is a paragraph with some text at the start of the article. This is just a sample but it has some mathematics like $x=y^2 +A_\mathrm{eff}$ and something else like $\hat{V}_\mathsf{L} = \sum_j \mathbf{r}_j^k$. Then it just starts repeating itself. This is a paragraph with some text at the start of the article. Then it just starts repeating itself. This is just a sample but it has some mathematics like $\chi = a_{d} \xi \zeta/\lambda = m_\mathrm{eff}d^2/(\hbar^2 \lambda)$ and something else like $r=|\vec{x}_A-\vec{x}_B|=\sqrt{(x_A-x_B)^2+(y_A-y_B)^2}$ or $d  \mathopen{}\left(  p+A(t)  \right)\mathclose{}  ^*$ . Then it just starts repeating itself.

This is another paragraph with more text just to show paragraph breaks. This is really just filler text but to be honest including a whole paragraph of Lorem Ipsum here seemed a bit too much. So instead here is a bit of filler text to flesh out the paragraph a bit.






\section{Here are some maths}
Here is a paragraph with a couple of citations: a paper \cite{Schulman-phys-rev},  a book \cite{Schulman}, and a miscellaneous item~\cite{TomScott}.

\lipsum[4]


Here's an equation with a label, \eqref{first-equation} and there's more equations further down. So, we have
\begin{equation}
V(\mathbf{x}_A,\mathbf{x}_B)=V(\vec{x}_A,\vec{x}_B)=d^2\frac{r^2-2\lambda^2}{(r^2+\lambda^2)^{5/2}},
\label{first-equation}
\end{equation}
being $d$ a letter, $\lambda$ a gathingammy, $r$ a letter in $r=|\vec{x}_A-\vec{x}_B|=\sqrt{(x_A-x_B)^2+(y_A-y_B)^2}$, with  $A$, $B$  labels. Moreover $\Lambda=\lambda/a$, and  $\chi = a_{d}/\lambda = m_{eff}d^2/(\hbar^2 \lambda)$, with $m_\mathrm{eff}=\hbar^2/2ta^2$, and $t$, are more maths expressions. So are $k=\sqrt{k_x^2+k_y^2}$ and $V_\mathrm{latt}(\vec r)= V_0\left(\sin^2(k_x x)+ \sin^2(k_y y)\right))$, and a displayed equation is 
\begin{equation}
\left(\hat{T}_A+\hat{T}_B+{\hat V}(\vec{x}_A,\vec{x}_B)\right)\Phi(\vec{x}_A,\vec{x}_B)=E\Phi(\vec{x}_A,\vec{x}_B).
\end{equation}

\lipsum[1]


\begin{align}
\partial_t \hat S(t,z) & = i\sqrt{d} \frac{\Gamma}{\Delta}\left( \Omega_+ \hat{\mathcal E}_+ + \Omega_- \hat{\mathcal E}_- \right) - \gamma \hat S \\
\partial_z \hat{\mathcal E}_+ (t,z) & = i \sqrt{d} \frac{\Omega_+}{\Delta} \hat{S}
\end{align}



\begin{figure}[b]
\centering
%\includegraphics[width=0.4\textwidth]{}
  \centering
  \fbox{
    \begin{minipage}[c][0.07\textheight][c]{0.4\textwidth}
      \centering{figure placeholder}
    \end{minipage}
  }
\label{fig-knot-table}
\figurecaption{%
Brief figure descrition%
}{%
Longer figure description.
}
\end{figure}



\lipsum[1-3]

Other displayed equations are
\begin{equation}
(\vec{\xi}_{\vec{K}}\cdot\vec{\hat{T}}_D+V(\vec{r}))\psi(\vec{r})=E\psi(\vec{r}),
\end{equation}
where $\vec{\xi }_{\vec{K}}=-2t(\cos(K_x a/2),\cos(K_y a/2))$ is some object, and  $\vec{\hat{I}}\cdot\vec{\hat{T}}_D\psi(\vec{r})=\sum_{i=x,y}\left(\psi(\vec{r}+\vec{\delta}_i)+\psi(\vec{r}-\vec{\delta}_i)\right)$, where $\vec{\delta}_i=a\hat{e}_{i}$, and also
\begin{equation}
\psi(\vec{r})=\frac{1}{N_x N_y}\sum_{\vec{q}}\psi(\vec{q})e^{i\vec{q}\cdot\vec{r}}.
\end{equation}
Here's a little bit of text to round out the paragraph after the equation and (probably) round out the page, so this needs to be a couple of lines long.

Here some text: \lipsum[7] 

Some text, then some more equations: these ones are
$$
E_{\vec{K},\vec{q}}=-4t\left(\cos(K_xa/2)\cos(q_xa)+\cos(K_ya/2)\cos(q_ya)\right)
$$ 
and 
\begin{equation}
(E-E_{\vec{K},\vec{q}})\psi(\vec{q})=\sum_{\vec{q'}}V(\vec{q}-\vec{q'})\psi(\vec{q'}). 
\end{equation}
To round out the paper, we now include a bunch of Lorem Ipsum paragraphs, some more math, more filler text, and call it a day.

\lipsum[1-2]


\section{More filler text}

\lipsum[1-6]

\section{Conclusions}

\lipsum[11-12]


\section*{Acknowledgements}
Here you thank all the people ever and everyone who ever gave you money.

%% To balance the columns on the last page.
\balance



%% Alternative bibliography style
%% \bibliographystyle{naturemag}


\bibliographystyle{arthur} 
\bibliography{references}{}
%% To get bibliography, as usual, run pdflatex -> bibtex -> pdflatex -> pdflatex.



\hfill % hacky fix for a bad-box warning caused by the balance package.

\end{document}














